\documentclass[reprint,amsmath,amssymb,aps,pre]{revtex4-1}
% Latex
\usepackage[english]{babel}
\usepackage[utf8]{inputenc}
\usepackage[T1]{fontenc}
% Xelatex
%\usepackage{polyglossia}
%\usepackage{fontspec}
%\setdefaultlanguage{english}
%\setmainfont{DejaVu Sans}
%\setsansfont{DejaVu Sans}
%\setmonofont{DejaVu Sans Mono}
\usepackage{bm}
\usepackage{cleveref}
\usepackage{xcolor}
\usepackage{algpseudocode}
\usepackage{graphicx}
\usepackage{subfigure}

\definecolor{light-gray}{gray}{0.95}
\newcommand{\code}[1]{\colorbox{light-gray}{\texttt{#1}}}
% When cleveref fails to do its job
\newcommand{\apref}[1]{Appendix \ref{#1}}

\begin{document}
%*Corresponding author: Kirill M. Gerke, Tel.: +79661877715, E-mail: kg@ifz.ru,
%Address: Schmidt Institute of Physics of the Earth of Russian Academy of
%Sciences, Bolshaya Gruzinskaya str. 10/1, Moscow, 123242, Russia
\author{Alexey I. Samarin}
\altaffiliation{Computational Mathematics and Cybernetics, Lomonosov Moscow
  State University, Moscow, 119991, Russia}
\affiliation{Schmidt Institute of Physics of the Earth of Russian Academy of
  Sciences, Moscow, 123242, Russia}
\author{Vasily Postnicov}
\affiliation{Schmidt Institute of Physics of the Earth of Russian Academy of
  Sciences, Moscow, 123242, Russia}
\author{Marina V. Karsanina}
\affiliation{Schmidt Institute of Physics of the Earth of Russian Academy of
  Sciences, Moscow, 123242, Russia}
\author{Efim V. Lavrukhin}
\altaffiliation{Computational Mathematics and Cybernetics, Lomonosov Moscow
  State University, Moscow, 119991, Russia}
\affiliation{Schmidt Institute of Physics of the Earth of Russian Academy of
  Sciences, Moscow, 123242, Russia}
\author{Kirill M. Gerke}
\email[E-mail:]{kg@ifz.ru}
\affiliation{Schmidt Institute of Physics of the Earth of Russian Academy of
  Sciences, Moscow, 123242, Russia}

\title{Robust surface correlation functions evaluation from discrete digital
  images}

\begin{abstract}
  What have we done?
\end{abstract}

\maketitle

% Write the content here!
\section{Background and motivation}
Correlation functions (CFs) are invaluable universal descriptors of
structure and in this context are utilized in a multitude of scientific
disciplines: material sciences \cite{Cecen}, rock physics, soil physics
\cite{Euras2012;PLoS_ONE;KarsaninaEJSS} and hydrology \cite{Schluter}, cosmology
\cite{oldJap} and food engineering \cite{Antonio}, to name just a handful. As
such, CFs were used to characterize the morphology \cite{tensorPRE} and
representativeness via correlation lengths \cite{Capek,Adler-Thovert}, compare
structures \cite{recons;REVpaper}, compress structural information
\cite{SciRep;Havelka}, describe structural dynamics \cite{Jiao_dynamo;PLoS_ONE},
extract features for deep learning \cite{Tahmasebi;Torq_SciRep;KarsaninaEJSS},
perform stochastic reconstructions
\cite{Adler;Y-T1998;EPL;Jiao;TahmasebiPRL;ourPRL} and fuse multi-scale images
\cite{SciRep;Jiao_multi;Karsanina2018;Tahmasebi2018}. Stochastic reconstruction
is a special topic of interest, as this approach allows to solve an inverse
problem and recover structure from a known set of correlation functions – and
this ability for recovery is the basis for majority of potential usages in the
list above. Early reconstruction techniques mainly involved two-point
probability $S_2$ function, but were improved to include lineal $L_2$ function
\cite{Y-T3D;Capek}, cluster $C_2$ function \cite{JiaoPNAS;JiaoChawla} and
surface-surface $F_{ss}$ function \cite{JiaoPNAS}. With the proper handling
\cite{EPL2} it is possible to incorporate numerous CFs into a reconstruction
procedure. Increasing the number (and order) of functions leads to higher
information content of the CFs set \cite{Gommes1-2}, thus, it allows to
reconstruct and perform stationarity/representativeness characterization
\cite{REV_paper;newPRL} for structures of any complexity.

There are two ways to obtain correlation functions for a given structure at hand
-- either measure them experimentally with the help of scattering intensity
\cite{Debye;Jiao_XCT}, or measured from images. The first approach suffers from
the limitation of CFs that can be obtained this way \cite{Gommes_notchord},
while the second one provides information with limited resolution or/and
resolution to field-of-view ratio \cite{SciRep}. Moreover, the most useful imaging
methods such as X-ray computed tomography (XCT) and scanning electron microscopy
(SEM) due to their underlying physical principles provide gray-scale images
which need to be segmented \cite{Schluter_review} into constituent phases before
CFs computation. It is important to note that such gray-scale images do not
represent phases spatial distribution, but, for example, X-ray spatial
attenuation or electron back-scattering intensities. However, if properly
segmented, high-resolution digital images do provide a possibility to compute
any correlation function. Experimental measurements using small angle scattering
(SAS) are traditionally used to evaluate $S_2$, but it is also possible to
relate scattering intensities to surface correlation functions
\cite{Ma_Torq+inthere}. It would be of great practical importance to obtain
surface correlation functions from both SAS and imaging (focused ion beam
milling combined with SEM allows to obtain the finest imaging resolutions
comparable to SAS and sample CFs from polished surfaces as opposed to surface
imaging \cite{FIB-SEM_paper}) to estimate coefficients for surface and bulk
scattering (see Eq.13 in \cite{Ma_Torq}). But there is a catch -- SAS possesses
close to infinite surface resolution as opposed to digital pixelized images with
inherent segmentation problems due partial volume effects, something we discuss
next.

Recently, Ma and Torquato \cite{Ma_Torq} laid foundation to precise surface
correlation function computations and showed the usefulness of these CFs
evaluation for numerous problems. They have implemented an elegant solution with
the help of infinite resolution random fields that allows finding an exact
intersection with the sampling line. In addition, they also showed that surface
CFs for digital images can be computed by converting integer fields (i.e.,
location of the phases) to float fields with the help of Gaussian filter
\cite{Ma_Torq}. Unfortunately, their solution is not readily applicable to the
majority of structure samples due to non-singularity in their chemical
constitution. The reason is the difference between the contour and real
interface between different phases within the material due to the partial volume
effects \cite{Wildenschild_Sheppard} for XCT imaging -- the presence of multiple
phases within the same pixel/voxel. In other words, the attenuation is a
function of both density and atomic number, which usually are distributed
non-uniformly below the XCT imaging resolution. Somewhat similar is also
relevant for SEM imaging, as secondary or back-scattered electrons are
effectively a convolution of partial signals coming from different depths
\cite{Bultreys_review}. The exact sub-voxel thresholding based on gray-scale
image (similar to the technique used in \cite{Ma_Torq}) is only available in
case it is monomineral. i.~e., the solid phase consists of a completely
chemically homogeneous substance that contrasts perfectly with air/vacuum filled
pore phase -- this is rarely the case for natural materials. All these and
additional problems (such as, for example, experimental noise and artifacts from
inverse Radon transform) arising during gray-scale image processing as related
to image segmentation were extensively discussed elsewhere
\cite{Safonov;Lavrukhin2021}. In other words, we still lack a robust and
computationally effective procedure to evaluate surface CFs from general 2D and
3D images of heterogeneous materials.

In this paper we build upon foundational work of Ma and Torquato \cite{Ma_Torq}
and develop a robust and efficient approach to compute surface correlation
functions from digital 2D and 3D images. The rest of the manuscript is organized
as follows: in \cref{sec:details} we provide all methodological details for
surface CFs computation including analytical solutions to verify the proposed
methodology and describe an image library for extensive testing of our
algorithms. \cref{sec:results} presents all major results of surface functions
evaluations. We discuss obtained results, including the effects of image
scaling, and outline future uses of surface functions within
\cref{sec:scaling}. The paper concludes with a summary in \cref{sec:summary}.

\section{Methodological details}
\label{sec:details}
\subsection{Correlation functions and definitions}
First, we introduce an indicator function $I^{(i)}(x)$, which describes the
affiliation between local points (pixels for 2D and voxels for 3D digitized
images) of structure under study. For a two-phase (or binary) system
(e.g. solid-pore) the indicator function will take the following form in each
location $x$ in the n-dimensional Euclidean space $\mathbb{R}^n$
\begin{equation*}
  I^{(i)}(x) = \left\{
  \begin{array}{ll}
    1 & \quad x \in V_i \\
    0 & \quad \text{otherwise}
  \end{array}
  \right.
\end{equation*}
where $V_i \subset \mathbb{R}^n$ is the region occupied by phase $i$. For
statistically homogeneous media the ensemble average of $I^{(i)}$ equals volume
fraction of a given phase. For binary media the following equality holds:
\begin{align*}
  \phi_{void} &+ \phi_{solid} = 1 \\
  \phi_i &= \langle I^{(i)}(x) \rangle
\end{align*}
In a similar fashion we can define an interface indicator function $M(x)$ which
provides interface area $s$ if averaged over the whole image:
\begin{align}
  M(x) &= |\nabla I^{(solid)}(x)| =|\nabla I^{(void)}(x)| \label{eq:interface} \\
  \langle M(x) \rangle &= s
\end{align}
The simplest, yet foundational correlation function is two-point probability
function $S_2$ which is defined as a probability to sample the ends of a line
segment within the same phase:
\begin{equation}
  S_2^{(i)}(x_1, x_2) = \langle I^{(i)}(x_1) I^{(i)}(x_2) \rangle \label{eq:twopoint}
\end{equation}
This equation can be further simplified for statistically homogeneous media, as
$S_2$ will dependent only on the relative displacement $r$:
\begin{equation*}
  S_2^{(i)}(x_1, x_2) = S_2^{(i)}(r)
\end{equation*}
Now, analogously to $S_2$ we can define surface-surface and surface-void
correlation functions:
\begin{align}
  F_{ss}(r) &= \langle M(x)M(x+r) \rangle \label{eq:fss} \\
  F_{sv}(r) &= \langle M(x)I^{(void)}(x+r) \label{eq:fsv} \rangle
\end{align}
The value of $S^{(i)}_2(r)$ at zero is a fraction of phase $i$ in a medium:
\begin{equation}
  S_2^{(i)}(0) = \phi_i \label{eq:s2porosity}
\end{equation}

By looking at \cref{eq:twopoint} and \cref{eq:fss}-\cref{eq:fsv} one can observe
some significant similarities between two-point probability and two-point
surface functions (see \cref{fig:scheme}). They will help us in
computations. However, let's first consider major differences. Firstly, unlike
$S_2$ in \cref{eq:s2porosity}, $F_{ss}(r)$ is not defined at $r=0$. $S_2$ can
be viewed as a autocorrelation of the image, i.~e., computing correlations
between shifted realization of the image -- something that can be used to
effectively compute $S_2$ with the help of fast Fourier transform (FFT) on
modern hardware, especially GPUs. If we apply the same analogy \cref{eq:fss}
can be considered as an intersection of the interface with itself for all
possible shifts. But for the correlation length zero this intersection is
technically infinity. While surface-void function is well defined at $r=0$, it
differs from $S_2$ in two significant aspects. Firstly, it mostly resembles
two-point cross-correlation function (as instead of autocorrelation we have to
compute correlation between the interface and the void phase). Secondly, one has
to keep in mind that cross-correlation is not commutative, so we get two
surface-void functions: correlation between an interface and a void phase and
correlation between a void phase and an interface. These considerations will be
very useful in understanding our computational framework. More detailed
information on two-point probability and surface correlation functions can be
found in comprehensive Torquato's book \cite{Torq_book}.

\begin{figure}[ht]
  \centering
  \includegraphics[width=0.9\linewidth]{images/scheme.png}
  \caption[]{A schematic depiction of a binary porous media (pores are shown in
    color) with examples of positive events for surface $F_{ss}$, $F_{sv}$ and
    two-point probability $S_2$ correlation functions. The zoomed in area
    represents the difference between the true ``continuous'' interface in
    between pore and solid phases with pixelized ``digital'' interface emerging
    due to limited resolution of digital images.}
  \label{fig:scheme}
\end{figure}

\subsection{Computation of correlation functions}
One can potentially apply a number of different computational approaches to
evaluate surface functions. We start by exploring the most computationally
straightforward algorithms and move towards more complex ones. \textbf{I cannot
 understand this.} The surface area
in \cref{eq:interface} has infinitesimal volume on digital images. This prevents
the usage of two methods to compute (cross-) correlations: scanning with line
segment and autocorrelation with FFT that are immediately applicable to digital
pixelized/voxelized images. The usage of interfaces between voxels ``as is''
leads to severe errors in $s$ and surface geometry, as known from application of
3D imaging for geometry and topology analysis \cite{AWR_PNM} or energy
minimization problems \cite{jagged_surfaces}. Another option would be to
describe the boundaries between voxels with some curves, e.~g., splines. This
way it would be possible to perform surface CFs sampling by line intersection as
described by Ma and Torquato \cite{Ma_Torq}. While an exact boundary can be
obtained for deterministic structures such as circle/sphere packings, splines
would provide only a very approximate solution for the boundaries of arbitrary
digitized structure due to the limit in resolution for a given image
\cite{Eusosoil2012}. In other words, contours extracted from digital images will
approach real boundaries between phases only when the spatial resolution
approaches infinity. But the same is also true for digital pixelized/voxelized
images. Thus, the options to compute surface correlation functions include
``continuous'' approach (such as implemented in \cite{Ma_Torq}) and ``digital''
approach (similar to computation of $S_2$ from digital images) as depicted in
\cref{fig:scheme}.

We adopt the ``digital'' approach, as it is clear that in the limit of infinite
resolution the ``continuous'' approach will not provide any advantages for
arbitrary XCT or SEM images.

%Во всем пейпере надо привести все A, M, Image и т.п. к единообразию тернимологии- а то
%кто в лес, кто по дрова…

\textit{Method 1}. The interface between binary phases is represented by a one
pixel thick contour (in two-dimensional case) or one voxel thick surface (in
three-dimensional case). This contour or surface is extracted from either solid
or void phase with the help of distance map transform with subsequent filtering
out of pixels/voxels with distances greater than $1$. Next, surface correlation
functions are computed as $S_2$ for the extracted interface using full
autocorrelation to obtain a correlation map or by scanning in predefined
directions. Such an extraction of the interface can be considered as a kind of
digitalization of the method based on dilation of spheres by Seaton and Glandt
\cite{SG1986}.

\textit{Method 2}. This approach computes correlation functions by scanning
along selected directions and extracting the slices from the array containing
the binary structure. The major difference from the previous method is how we
extract the boundary between phases from the digital image. Given a multiphase
image $A$ we reduce it to a two-phase image $A'$ by applying the indicator
function $I$ to select the phase of interest: $A' = I(A)$. We use Sobel operator
to detect edges of the image $A'$, later denoted as $\nabla A$:
$\nabla A' = S(A')$. We calculate $\|\nabla A'\|$ element-wise ($\|\cdot\|$
denotes the usual norm on Euclidean space). This norm can be interpreted as the
probability that a point in $A$ belongs to the interface. In a digitized image
$\|\nabla A'\|$ can be represented as an array of floating point numbers having
the same shape as an original image $A$. The next step is to cut one-dimensional
slices from the array $\|\nabla A' \|$ in multiple predefined directions. In our
code, the directions include two orthogonal and two diagonal directions for 2D
images, and three orthogonal and ten diagonal directions in 3D. For each slice
$a_{ij}$ ($i$ being the number of a slice and $j$ being the number of an element
in that slice) we compute two sequences:
\begin{align}
  S^{(ss)}_{ik} &= (a_i \star a_i)_k \label{eq:fss-success} \\
  N_{ik} &= \text{\# of subslices with the length $k+1$ in the slice $i$} \label{eq:divisor}
\end{align}
where $\star$ denotes a discrete cross-corellation of two sequences with the
length $N$:
\begin{equation*}
  (a\star b)_i = \sum_{j=0}^N a_j b_{i+j \mod N}
\end{equation*}
The surface-surface correlation function for the correlation length k can be
computed using the formula:
\begin{equation*}
  F_{ss}(k) = \frac{\sum_i S^{(ss)}_{ik}}{\sum_i N_{ik}} \qquad k \ge 0
\end{equation*}

The surface-void correlation function $F_{sv}$ can be computed in a similar
way. At first, calculate $\|\nabla A'\|$ in the same way as for $F_{ss}$. Then
cut slices $a_{ij}$ from the array $\|\nabla A'\|$  and $b_{ij}$ from the same
positions from the array $B = I^{(void)}(A)$. The equation \cref{eq:fss-success}
becomes
\begin{equation}
  S^{(sv)}_{ik} = (a_i \star b_i)_k \label{eq:fsv-success}
\end{equation}
$F_{sv}(k)$ can be computed using \cref{eq:fsv-success} and \cref{eq:divisor} in
the following way:
\begin{equation*}
  F_{sv}(k) = \frac{\sum_i S^{(sv)}_{ik}}{\sum_i N_{ik}} \qquad k \ge 0
\end{equation*}

In practice we use the discrete Fourier transform to calculate a
cross-correlation $a \star b$ as $F^{-1} [F(a) \cdot \overline{F(b)}]$.

Here follow step-by-step algorithms to compute surface-surface and surface-void
correlation functions:
\begin{algorithmic}[1]
  \Procedure{surfsurf}{$a, phase, direction$}
  \Comment{input image $a$, phase $phase$, direction vector $direction$}
  \State $S \gets [0,0..]$
  \State $N \gets [0,0..]$
  \State $a' \gets I^{(phase)}(a)$
  \Comment{Apply indicator function element-wise.}
  \State $ind = \|S(a')\|$
  \Comment{Interface detection with Sobel filter.}
  \For{$slice \in \{ \text{Slices from } ind \parallel direction\}$}
  \State $S \gets S + F^{-1}[F(slice) \cdot \overline{F(slice)}]$
  \Comment Element-wise addition
  \State $N \gets N + N'_k$ \quad where $N'_k$ is a list with number of subslices of length $k+1$ in $slice$.
  \EndFor
  \State $F_{ss} \gets S/N$
  \Comment Element-wise division.
  \State \textbf{return} $F_{ss}$.
  \EndProcedure
\end{algorithmic}

\begin{algorithmic}[1]
  \Procedure{surfvoid}{$a, phase, direction$}
  \Comment{input image $a$, phase $phase$, direction vector $direction$}
  \State $S \gets [0,0..]$
  \State $N \gets [0,0..]$
  \State $a' \gets I^{(phase)}(a)$
  \Comment{Apply indicator function element-wise.}
  \State $void \gets I^{(void)}(a)$
  \State $ind = \|S(a')\|$
  \Comment{Interface detection with Sobel filter.}
  \For{$s_1 \in \{ \text{Slices from } ind\} \quad \text{and} \quad s_2 \in \{\text{Slices from } void\}, s_1 \parallel s_2 \parallel direction$}
  \State $S \gets S + F^{-1}[F(s_1) \cdot \overline{F(s_2)}]$
  \Comment Element-wise addition
  \State $N \gets N + N'_k$ \quad where $N'_k$ is a list with number of
  subslices of length $k+1$ in $s_1$ or $s_2$.
  \EndFor
  \State $F_{sv} \gets S/N$
  \Comment Element-wise division.
  \State \textbf{return} $F_{sv}$.
  \EndProcedure
\end{algorithmic}

\textit{Method 3}.

%accurate approximation of the interface. For simplicity, all methodological details are provided as
%related to 2D images; extension to 3D is straightforward and is provided within our code example in
%Supplementary Materials. In order to calculate (, ) = |∇(, )|, we first estimate the
%gradient (, ) using the Sobel filter kernel. As a result, we obtain two matrices containing
%gradients in the direction of the first and second dimension  `,  `. Then,
%
%                                                                                                       
%(, ) =  `2(, ) +  `2(, ).
%
%
%Calculation of void indicator function is even simpler:
% (, ) = 1 − (, ).
%
% 
%Cross correlation function  () = 〈(1)()(2)( + )〉 = < () ( + ) > can be
%
%                                                      
%easily computed using fast Fourier transform [link to paper]. First, we calculate Fourier transform
%of  and :  = (),  = (). Then we take the complex conjugate of each element of
%the matrix F and multiply it by the corresponding element of the matrix G:  = * * . Taking the
%inverse Fourier transform from the resulting matrix, we get an unnormalized cross-correlation
%matrix () = (). Let () denote the number of all possible points within the image, the
%distance between which is equal to r, then by dividing the matrices () and ()elementwise, we
%obtain the cross-correlation matrix.
%
%     In case the boundary conditions are periodic, () does not depend on r and is equal to the
%number of pixels on the image. For non-periodic boundary conditions, () is equal to the number
%of pixels whose coordinates are greater than . Since the FFT assumes periodic boundary conditions
%by default, it is required to additionally prepare the image when the boundaries are non-periodic. To
%do this, the image is doubled in each dimension, filling new pixels with zeros. After that, the
%algorithm is applied without any changes until the matrix  is obtained, from which, as an
%additional step, the matrix of the size of the original image is cut out.
%
%     Taking the function (, ) instead of , , we get  (), and if we take (, ) instead of 
%
%                                                                                                    
%
%and ()(, ) instead of , we obtain  (). More precisely, we have counted functions only for
%
%                                                                        
%
%non-negative r. To calculate functions when some of the components of r are negative, it is enough
%to reverse the image in the corresponding dimensions and perform all the original steps of the
%algorithm.
%
%     Sometimes one only needs to calculate a function in a given direction. In this case, after
%obtaining  and (), it is necessary to cut 1D slices along the required direction, calculate the
%one-dimensional cross-correlation, add the results obtained and divide by .
%
%     В методы Леши надо бы упомянуть конверсию мультифазки в двух как описано у Васи?
%А то как то странно смотрится все это…
%

All methods and their consecutive stages are shown in \cref{fig:stages}. They
are also a part of \code{CorrelationFunctions.jl} package \cite{ourpapaer} for
Julia programming language that allows to compute all classical CFs described in
Torquato’s book \cite{Torq_book} from digital images.

\subsection{Analytical solutions for surface CFs}
For Poisson disks and spheres as shown in the previous subsection one can derive
exact analytical surface-surface and surface-void functions. For overlapping
disks and spheres with radius $R$ and centers generated by Poisson point process
with parameter $\lambda$ surface-surface function $F_{ss}(r)$ and surface-void
function $F_{sv}(r)$ we have (see the derivation of these formulas in
\cref{ap:overlapping-disks}.
\begin{align}
  F_{sv}(r) &= \left\{
  \begin{array}{ll}
    2(\pi - B)R \lambda S_2(r) & \quad r<2R \\
    2\pi R \lambda S_2(r) & \quad \text{otherwise}
  \end{array} \right. \label{eq:fsv_final} \\
  F_{ss}(r) &= \left\{
  \begin{array}{ll}
    \frac{(2(B-\pi)R\lambda)^2Ar + 4\sqrt{A}R^2\lambda}{Ar}S_2(r) & \quad \text{otherwise}
  \end{array} \right. \label{eq:fss_final}
\end{align}

where $S_2$ is the regular two-point correlation function and
\begin{align*}
  A &= 4R^2 - r^2 \\
  B &= \arccos(\frac{r}{2R})
\end{align*}

For 3D spheres the relationships are simpler and readily available in the
literature \cite{Torq_book}\cite{Ma_Torq}:
\begin{align*}
  F_{sv}(r) &= \left\{
  \begin{array}{ll}
    4\pi R^2\lambda(\frac{1}{2} + \frac{r}{4R})S_2(r) & \quad r<2R \\
    4\pi R^2\lambda S_2(r) & \quad \text{otherwise}
  \end{array} \right. \\
  F_{ss}(r) &= \left\{
  \begin{array}{ll}
    {(4\pi R^2 \lambda (\frac{1}{2} + \frac{r}{4R}))^2 + \frac{2\pi R^2 \lambda}{r}} & \quad r<2R \\
    (4\pi R^2 \lambda)^2 S_2(r) & \quad \text{otherwise}
  \end{array} \right.
\end{align*}

These analytical solutions are used to verify the accuracy of computations for
Poisson disks/spheres.

\section{Application to synthetic and real binary 2D/3D images}
\label{sec:results}
\subsection{Comparisons against analytical solutions}
To evaluate the accuracy of surface CFs computations the most straightforward
way is to compare them against analytical solutions. We begin by considering a
single circle, and then the realization of the Poisson point process and draw
spheres of radius $R$ with centers at those locations. Starting with an image
with resolution of 4096x4096 pixels, we then downscale it in 4, 16, and 64 times
with the help of bicubic interpolation \cite{mexicans}. It is easy to show that
assuming $a > 0$:
\begin{align*}
  F_{ss}(a \mathbb{x}, a(\mathbb{x} + \mathbb{r})) &= a^2 F_{ss}(\mathbb{x},
  \mathbb{x} + \mathbb{r}) \\
  F_{sv}(a \mathbb{x}, a(\mathbb{x} + \mathbb{r})) &= a F_{sv}(\mathbb{x},
  \mathbb{x} + \mathbb{r})
\end{align*}

Assuming that our images represent homogeneous and isotropic media, we calculate
$F_{ss}(r)$ and $F_{sv}(r)$ for each original or rescaled image and multiply it
by coefficient $a^2 = (L_{scaled}/L_{orig})^2$ and $a = L_{scaled}/L{orig}$
respectively, where $L_{scaled}$ is the side of the rescaled image and
$L_{orig}$ is the side of the original image. The resulting scaled surface CFs
as computed using the Method 2 are shown in \cref{fig:scaling}. Note that
Method 2 and Method 3 produce the same results if computed along the same
direction.

\begin{figure*}[ht]
  \centering
  \subfigure[Surface-surface]{
    \includegraphics[width=0.475\linewidth]{images/plot-ss-balls.png}
    \label{fig:fss-scaling}}
  \hfill
  \subfigure[Surface-void]{
    \includegraphics[width=0.475\linewidth]{images/plot-sv-balls.png}
    \label{fig:fsv-scaling}}
    \caption[]{The comparison of analytical and computed surface correlation
      functions for the Poisson circles for the original 4096x4096 pixels
      image. Computed CFs were obtained for different image resolutions as shown
      in the legend.}
    \label{fig:scaling}
\end{figure*}

We can observe two effects when downscaling the original image. The
first effect is that $F_{ss}(r)$ and $F_{sv}(r)$ for downscaled images lack in
details (i.e., they are less ``noisy''). This can be easily explained by that the
interface between phases that becomes ``simpler'' when resolution decreases. The
second and much more important effect is that both correlation functions become
largely underestimated with the decline in resolution. This latter effect is in
general connected to the fact that the interface in digital images has
non-negligible thickness -- i.~e. the difference between real ``continuous''
interface versus ``digital'' interface \cref{fig:scheme}. It is logical to
assume that with increasing spatial resolution the influence of this thickness
will diminish. This is exactly what we observe on \cref{fig:scaling} where
increasing disk discretization leads to convergence with analytical solution;
the accuracy of the computed CFs is almost perfect for discretization of XX
pixels for each disk. A natural question arises: ``Is there a criterion of image
quality (resolution) that allows to predict the quality of surface CFs
evaluation from digital images?''. Turns out there is a possibility to establish
such empirical criterion based on spectral analysis of input images and
\cref{sec:wtf} provides necessary details.

Now consider a situation when an image is obtained by taking samples of a function
$f: \mathbf{R}^n \rightarrow \left\{0, 1\right\}$. The samples are taken at a regular grid
which covers the range $[0, 1]^n$  with interval $\Delta$ between samples. The
resulting image has then $1/\Delta + 1$ pixels in each dimension. An example of
$f$ is a thresholded value noise function, and an example of image generated by
sampling this function is on \cref{fig:noise}.

\begin{figure}[ht]
  \centering
  \includegraphics[width=0.9\linewidth]{images/noise.png}
  \caption[]{Example of thresholded value noise used for study of sampling
    effects.}
  \label{fig:noise}
\end{figure}

If we calculate surface-surface and surface-void correlation functions for these
images and scale them as described above, we will converge to some ``true''
correlation functions for continuous function $f$.

\begin{figure*}[ht]
  \centering
  \subfigure[Surface-surface]{
    \includegraphics[width=0.475\linewidth]{images/plot-ss-noise.png}
    \label{fig:fss-scaling-noise}}
  \hfill
  \subfigure[Surface-void]{
    \includegraphics[width=0.475\linewidth]{images/plot-sv-noise.png}
    \label{fig:fsv-scaling-noise}}
    \caption[]{Surface-surface and surface-void correlation functions for
      thresholded value noise function sampled with different resolutions.}
    \label{fig:scaling-noise}
\end{figure*}

\subsection{The choice of filter’s kernel to extract the interface (gradient)}
There are many different kernels to evaluate gradients, but the Sobel kernel
performed slightly better \cref{fig:wtf2} compared to majority of other kernels
we have explored: Ando, Scharr, Bickley, and Prewitt filters. Only the Ando’s
filter with size of 5x5 \cite{ando_2000} outperformed the Sobel filter in terms
of error minimization for 2D circles surface CFs computation. Nonetheless, we
chose Sobel filter due to its simplicity, faster computations and robust
implementation of the 3D kernel in all major libraries (where Ando is only a 2D
filter). Gauss filter is not improving \cref{fig:wtf3}.

\subsection{Criteria for accurate evaluation of surface CFs from discrete images}
Let us define one-dimensional forward ($F$) and inverse ($F^{-1}$) Fourier
transforms as:
\begin{align}
  \hat{f}(z) &= F[f](z) = \frac{1}{\sqrt{2\pi}}\int_{-\infty}^{\infty} f(x)
  e^{-i\pi xz} dx \label{eq:fourier-forward} \\
  f(x) &= F^{-1}[\hat{f}](z) = \frac{1}{\sqrt{2\pi}}\int_{-\infty}^{\infty} \hat{f}(z)
  e^{i\pi xz} dz \label{eq:fourier-backward}
\end{align}

Here $f(x)$ and $\hat{f}(z)$ can be thought of as representations of a signal in
the time domain and the frequency domain, respectively. Both equations
\cref{eq:fourier-forward} and \cref{eq:fourier-backward} preserve norm on $L_2$:
$(f, f) = (\hat{f}, \hat{f})$. It is said that Fourier transform preserves
``energy'' of the signal. According to Riemann-Lebesgue lemma \cite{bookHilb}
energy of any signal ``concentrates'' around low frequencies. A measure of how
much energy concentrated in low frequencies (for some definition of low
frequencies) is a key to the understanding of the problem of correctness of our
method. The Shannon sampling theorem \cite{bookHilb} states that it is possible
to reconstruct a band-limited signal $f(x)$ whose frequencies lie in the range
$[0, f]$ from a sequence of samples $\left\{f\right\}$ if the sampling rate is
no less than $2f$. We introduce a parameter $C_a$:
\begin{align*}
  f_0(x) &= f(x) - \langle f(x) \rangle \\
  C_a &= \frac{\int_{-a\omega}^{a\omega} |\hat{f_0}(z)|^2 dz}{\int_{-\omega}^{\omega} |\hat{f_0}(z)|^2 dz}
\end{align*}
where $\omega = 2\pi f$ and $\langle f(x) \rangle$ is the mean value of $f(x)$
over its domain. The criterion for correctness is then:
\begin{equation*}
  C_a > 1 - \xi
\end{equation*}
for some $a$ and $\xi$. This criterion tells us exactly how much energy in
$f_0(x)$ is concentrated in a low frequency range $[0, af]$ compared to the
whole range $[0, f]$. In our work we propose $a = 0.5$ and $\xi = 0.05$ that are
a strict criterion that is based on results of section XXX. For the images of
Poisson circles from \cref{fig:scaling} their discretizations and $C_{0.5}$ are
shown in \cref{fig:disks-res}. We immediately observe that the top left image
(with resolution 4096x4096 and surface CFs very close to analytical solution)
satisfies our $C_{0.5}$ criterion perfectly. The top right image fails to
satisfy the criterion, but only to some small degree ($0.9185 \approx 0.95$).
The other two downscaled images completely fail to satisfy the criterion. As we
can see in conjunction from \cref{fig:scaling} and \cref{fig:disks-res}, the
criterion correctly selects images which are suitable for calculation of the
correlation functions using Method2 and Method3. The criterion $C_{0.5}$ decays
fast with the decline in image resolution as shown in \cref{fig:crit-plot}.

\begin{figure}[ht]
  \centering
  \includegraphics[width=0.9\linewidth]{images/plot-criterion.png}
  \caption[]{The decay of criterion $C_{0.5}$ as a function of the size $s$ of
    the downscaled image.}
  \label{fig:crit-plot}
\end{figure}

\begin{figure*}[t]
  \centering
  \subfigure[Resolution 4096x4096, disk radius 61.44 pixels, $C_{0.5} = 0.9593$]{
    \includegraphics[width=0.475\linewidth]{images/disks-0015-5e-5-4096.png}
    \label{fig:disks-4096}}
  \hfill
  \subfigure[Resolution 1024x1024, disk radius 15.36 pixels, $C_{0.5} = 0.9185$]{
    \includegraphics[width=0.475\linewidth]{images/disks-0015-5e-5-1024.png}
    \label{fig:disks-1024}}
  \vskip\baselineskip
  \subfigure[Resolution 256x256, disk radius 3.84 pixels, $C_{0.5} = 0.8356$]{
    \includegraphics[width=0.475\linewidth]{images/disks-0015-5e-5-256.png}
    \label{fig:disks-256}}
  \hfill
  \subfigure[Resolution 64x64, disk radius 0.96 pixels, $C_{0.5} = 0.6549$]{
    \includegraphics[width=0.475\linewidth]{images/disks-0015-5e-5-64.png}
    \label{fig:disks-64}}
  \caption[]{The influence of imaging resolution on the quality of surface
    correlation functions computations (as shown in \cref{fig:scaling}) based on
    the criterion $C_{0.5}$.}
  \label{fig:disks-res}
\end{figure*}

\subsection{Application to binary images of porous media}
After the verification of our computational approach based on analytical
solution and establishing the criterion $C_{0.5}$ we now have enough tools to
evaluate to compute surface CFs for different porous media images, including
real XCT and SEM images.

\section{Discussion and outline}
Exact vs. for-reconstruction computation of surface CFs

Our way it is easy to implement higher-order statistics computations, such as
multiple-point surface-surface correlation functions.

Discuss super-resolution to improve accuracy of surface functions computation
Explain that ``digital'' is better than ``continuous''.

\section{Conclusion}
Our summary says

\section{Acknowledgments}
This research was supported by the Russian Science Foundation grant
17-17-01310.

Collaborative effort of the authors within the FaT iMP (Flow and Transport in
Media with Pores) research group (www.porenetwork.com) and used some of its
software. We sincerely thank Prof. Salvatore Torquato for directing us to a
derivation of analytical formulas for 2D Poisson disks.

\appendix
\section{Derivation of analytical solutions for 2D Poisson disks}
\label{ap:overlapping-disks}
Analytic representation of surface-surface and surface-void correlation
functions for overlapping balls with centers generated by Poisson point process
is well known \cite{Torq_book}:
\begin{align}
  F_{sv}(r) &= -\lim_{a_1 \rightarrow R} \frac{\partial}{\partial a_1}
  e^{-\lambda S_{tot}(r, a_1, R)} \label{eq:fsv-disks} \\
  F_{ss}(r) &= -\lim_{a_1, a_2 \rightarrow R} \frac{\partial}{\partial a_1}
  \frac{\partial}{\partial a_2} e^{-\lambda S_{tot}(r, a_1,
    a_2)} \label{eq:fss-disks} \\
\end{align}
Here $S_{tot}(r, a_1, a_2)$ refers to a common volume of two n-dimensional balls
of radii $a_1$ and $a_2$ with a distance $r$ between their centers and $\lambda$
is a parameter of Poisson process. For two-dimensional disks we have the
following expression for $S_{tot}(r, a1, a2)$:
\begin{equation}
  S_{tot}(r, a_1, a_2) = \pi a_1^2 + \pi a_2^2 - S_{int}(r, a_1, a_2) \label{eq:total}
\end{equation}
where $S_{int}(r, a_1, a_2)$ is a common area of two disks, being equal to:
\cite{Math_stack_link}
\begin{align}
  S_{int}(r, a_1, a_2) =&  a_1^2 \arccos(\frac{r^2+a_1^2-a_2^2}{2a_1r}) + \\
  & a_2^2 \arccos(\frac{r^2+a_2^2-a_1^2}{2a_2r}) - \\
  & \frac{\sqrt{Y}}{2} \label{eq:intersection}
\end{align}
when $r<2R$ and zero otherwise. $Y$ in \cref{eq:intersection} is
\begin{equation*}
  Y = (-r+a_1+a_2)(r+a_2-a_1)(r+a_1-a_2)(r+a_1+a_2)
\end{equation*}
Substituting \cref{eq:intersection} into \cref{eq:total} and then
\cref{eq:total} into \cref{eq:fss-disks} we obtain the expression
\cref{eq:fss_final} for $F_{ss}(r)$.

Similarly substituting \cref{eq:intersection} into \cref{eq:total} and then
\cref{eq:total} into \cref{eq:fsv-disks} we obtain the expression
\cref{eq:fsv_final} for $F_{sv}(r)$.

\appendix
\section{Implementation of surface function computations}
The code used to compute surface correlation functions in this manuscript was
written in Julia language and available as a Jupyter notebook in Supplementary
Materials. It uses \code{CorrelationFunctions.jl} package developed by our group
\cite{CorrFunc.jl_paper} that allows efficient computation of surface and other
correlation functions using both CPU and GPU architectures.

\appendix
\section{Surface-surface correlation function for one disk}
The analytical representation of the surface-surface correlation function for
one disk is as follows:
\begin{equation*}
  F_{ss}(r) = \lim_{a_1, a_2 \rightarrow R} \frac{\partial}{\partial a_1}
  \frac{\partial}{\partial a_2} S_{tot}(r, a_1, a_2)
\end{equation*}
Following the same steps as in \cref{ap:overlapping-disks} we get a simple
formula:
\begin{equation*}
  F_{ss}(r,R) = \left\{
  \begin{array}{ll}
    \frac{4R^2}{r\sqrt{4R^2-r^2}} & \quad r < 2R \\
    0 & \quad \text{otherwise}
  \end{array} \right.
\end{equation*}
The resulting expression depends only on $t = r / 2R$:
\begin{equation*}
  F_{ss}(t) = \left\{
  \begin{array}{ll}
    \frac{1}{t\sqrt{1-t^2}} & \quad t < 1 \\
    0 & \quad \text{otherwise}
  \end{array} \right.
\end{equation*}
You can see that as $t$ tends to $0$ on the right or $1$ on the left,
$F_{ss}(t)$ tends to infinity.

\onecolumngrid
\bibliography{habib}
\bibliographystyle{plain}
\twocolumngrid

\end{document}
